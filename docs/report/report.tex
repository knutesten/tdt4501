\documentclass[11pt,twoside,a4paper]{report}
\usepackage[utf8]{inputenc}
\usepackage{hyperref}
\usepackage{listings}
\usepackage{color}
 
\definecolor{dkgreen}{rgb}{0,0.6,0}
\definecolor{gray}{rgb}{0.5,0.5,0.5}
\definecolor{mauve}{rgb}{0.58,0,0.82}
 
\lstset{ 
  language=Java,                % the language of the code
  basicstyle=\footnotesize,           % the size of the fonts that are used for the code
  numbers=left,                   % where to put the line-numbers
  numberstyle=\tiny\color{gray},  % the style that is used for the line-numbers
  stepnumber=1,                   % the step between two line-numbers. If it's 1, each line 
                                  % will be numbered
  numbersep=5pt,                  % how far the line-numbers are from the code
  backgroundcolor=\color{white},      % choose the background color. You must add \usepackage{color}
  showspaces=false,               % show spaces adding particular underscores
  showstringspaces=false,         % underline spaces within strings
  showtabs=false,                 % show tabs within strings adding particular underscores
  frame=single,                   % adds a frame around the code
  rulecolor=\color{black},        % if not set, the frame-color may be changed on line-breaks within not-black text (e.g. commens (green here))
  tabsize=2,                      % sets default tabsize to 2 spaces
  captionpos=b,                   % sets the caption-position to bottom
  breaklines=true,                % sets automatic line breaking
  breakatwhitespace=false,        % sets if automatic breaks should only happen at whitespace
  title=\lstname,                   % show the filename of files included with \lstinputlisting;
                                  % also try caption instead of title
  keywordstyle=\color{blue},          % keyword style
  commentstyle=\color{dkgreen},       % comment style
  stringstyle=\color{mauve},         % string literal style
}

\usepackage{graphicx}
\graphicspath{{./images/}}

\begin{document}
\begin{titlepage}

\newcommand{\HRule}{\rule{\linewidth}{0.5mm}} % Defines a new command for the horizontal lines, change thickness here

\center % Center everything on the page
 
%----------------------------------------------------------------------------------------
%	HEADING SECTIONS
%----------------------------------------------------------------------------------------

\textsc{\LARGE Norwegian Univeristy of Science and Technology}\\[1.5cm] % Name of your university/college
%\textsc{\Large Major Heading}\\[0.5cm] % Major heading such as course name
%\textsc{\large Minor Heading}\\[0.5cm] % Minor heading such as course title

%----------------------------------------------------------------------------------------
%	TITLE SECTION
%----------------------------------------------------------------------------------------

\HRule \\[0.4cm]
{ \huge \bfseries Falldetection Prototype}\\[0.4cm] % Title of your document
\HRule \\[1.5cm]
 
%----------------------------------------------------------------------------------------
%	AUTHOR SECTION
%----------------------------------------------------------------------------------------

\begin{minipage}{0.4\textwidth}
\begin{flushleft} \large
\emph{Authors:}\\
Dean \textsc{Lozo}\\ % Your name
Knut \textsc{E. M. Nekså}\\ % Your name
\end{flushleft}
\end{minipage}
~
\begin{minipage}{0.4\textwidth}
\begin{flushright} \large
\emph{Supervisor:} \\
Dag \textsc{Svanæs} % Supervisor's Name
\end{flushright}
\end{minipage}\\[3cm]

% If you don't want a supervisor, uncomment the two lines below and remove the section above
%\Large \emph{Author:}\\
%John \textsc{Smith}\\[3cm] % Your name



%----------------------------------------------------------------------------------------
%	LOGO SECTION
%----------------------------------------------------------------------------------------

\includegraphics{ntnuEnglish.png}\\[1.5cm] % Include a department/university logo - this will require the graphicx package

%----------------------------------------------------------------------------------------
%	DATE SECTION
%----------------------------------------------------------------------------------------

{\large \today}\\[3cm] % Date, change the \today to a set date if you want to be precise

%----------------------------------------------------------------------------------------

\vfill % Fill the rest of the page with whitespace

\end{titlepage}

%\maketitle

% Top matter end

\begin{abstract}

\end{abstract}

\tableofcontents

\chapter{Introduction}
%The Introduction is your thesis in a nutshell. Again, the organization can vary, but a standard introduction includes the following sections:

\section{Purpose}
The purpose of the project is to explore the possibilities of creating bio-feedback systems through the use of a smartphone and COTS products, mainly game controllers. The purpose of the bio-feedback system will be to improve the stability and balance of the user through peripherals connected to the phone. Audio and vibration will be used to provide feedback to the user about his stability. In the case of a fall the system will send out alerts to inform personnel of the incident.

\section{Motivation}
%Brief description of the research domain and the problem that one wants to address. It should tell the reader why working on this project is worth doing.
Falls are a major health hazard among the elderly population (age > 65) \cite{fallsHealthHazard}, in addition to being an obstacle for physical activity and independent living. Through physical activity the elderly may improve their quality of life and prevent future disabilities\cite{physicalActivity}. A third of all elders that experience a fall develop a fear of falling \cite{fearOfFalling}. Fear of falling causes general anxiety and avoidance of physical activity. Long term consequences result in social isolation, physical deterioration and reduced quality of life.\cite{physicalAvoidance} %Skrive noe om hvor mye det koster staten å passe på de eldre?

Between 30\% and 60\% of the elder population will experience at least one fall per year, and 10\% - 20\% of these will result in an injury, hospitalization or death \cite{fallStatistics}. For the independent elder population it is even more crucial that a fall can be avoided and the proper authorities can be alerted if a fall does occur. In the case of a serious fall a slow response time might increase the likelihood of permanent damage or death\cite{personHomeDeath, dangerousFallHome}.


\chapter{State of the Art}
%This chapter provides an overview of the literature. It positions your work with respect to work already done by others. 

Several attempts have been made to create fall detection systems using smart phone technology \cite{iFall, semiSupervisedFallDetection, mobilePhoneBasedFallDetection, detectionOfFalls}, all of these studies show positive results. The current trend shows that smart phones are becoming more affordable \cite{find_some_data_here} and with the correct software these phones can notify healthcare personnel about falls, giving the elderly an extra safety.

Another approach is to try preventing the falls from happening all together by notifying the user that he or she is moving in an unstable manner. Multiple studies show that using biofeedback systems to inform the user about unstable walk helps prevent falls \cite{multiModualBiofeedback, vibrotactileBiofeedback, vibrotactileTiltFeedback}. % KANSKJE SKRIVE NOE OM AT SENSORENE ER DYRE ELLER TUNGVINTE Å BRUKE I HJEMMET?
Also mobile versions of such systems have been researched with promising results \cite{fallPrevention}.

A limitation for many of the previously discussed solutions is that they only make use of the embedded accelerometers in the smart phone. Increasing the number of sensors and their positioning can make the fall detection application more accurate \cite{fallDetectionWithExtraSensors}.

In this study the goal is to examine if using cheap external sensors can give more accurate and additional sensor data to help make fall detection/prevention applications more effective. The sensors examined will be the Wii Remote and the Sony Move controller. These are cheap mass produced game controllers with embedded sensors and Bluetooth support. Using Bluetooth wireless technology the sensors can be connected to the smartphone and stream supplementary acceleromter and gyroscope data.

The Google Play Store has 3 applications that allow you to connect a WiiRemote to your Android Device. All of these applications focus only on the buttons, and ignore sensor data \cite{wiimoteController, simpleWiiController}. The applications seem unstable and are far from user friendly, but they are the only of its kind out there.

\section{Research questions}
%What are the questions you are answering with your project? Normally, you specify a main question and related sub-questions. Remember that at the end you have to demonstrate you have answered to the stated questions. It is not uncommon that the questions are changed during the project, but it is important to be as explicit as possible and as early as possible with research questions since they help you to focus.
RQ1: Can a mobile bio-feedback system be created using smartphones and gaming peripherals.

RQ1.1: Can motion sensing game controllers be used to gather additonal and more accurate data for fall related applications on the smartphone? What is the optimal location for the controller to be located?

RQ1.2: Is it feasible to have game controllers connected to the smartphone? How is the battery life, how close does the phone have to be at all times?

RQ1.3: How many bluetooth devices can be connected to the phone?


\section{Research method}
%How the research is conducted. In the previous section you say what you are doing. Here you specify how. The choice of a research method is strictly connected to the type of questions you want to answer.
There is very little to no research done on connecting smartphones and gaming peripherals together. Therefore a case study will be preformed. A case study was chosen because it was the only choice. We can not answer any of our research questions without having an application or system to test it out on. Currently no such system exists, so one must be created.

The case study will be split into two parts. The first part will be a qualitative technical study. An application will be created for the Android phone which will function as a hub for the bio-feedback system. This study will focus on any limitations, or particular challenges that the application might have from a technical standpoint.

The second part will be a quantitative study, where the viability of the system will be tested. Accuracy, battery life, range, stability are all factors that will determine if the system is reliable enough to be used out in the field, or as a technical foundation for a more user friendly bio-feedback system.



\section{Android OS}
Should we write something about the Android OS?

\section{Java}
Is it neccessary to write something aobut Java?

\section{PS3 Move}

\subsection{Motion Controller Hardware}
The PS Move motion controller contains advanced motion sensing, making it an ideal peripheral for a bio-feedback system. It features motion tracking through a three axis accelerometer, a three axis angular rate sensor and provides location tracking through the built in magnetometer. The built in vibration could be used to provide feedback to the user \cite{psMoveTech}.

\subsection{PS Move API}

The PS Move API \cite{PSMoveAPI} is written in C, but contains bindings for various languages, including Java and C\# which are both languages used in smartphone application development. The API explicitly mentions that it runs on Android devices. This is a truth with modifications. It will not run out of the box on an Android device, in addition restructuring and heavy modification of the Android device is required.  The Android OS runs on top of a modified Linux kernel, this kernel does not contain the necessary libraries and drivers in order for the API and Motion controller connectivity to function properly. %Does this need citation?
The next step would be to compile the C API into a shared library using the Android NDK, and use the shared library Java bindings in the Android Java code.

Everything is possible with enough time, but given the time constraint and amount of time required to get this running it was decided that this was not a path worth pursuing for this project. With some much work required it would be simpler to run Ubuntu off an Android device if this becomes available in the future. \cite{ubuntuAndroid}

\section{Wii Remote with Motion Plus}

\subsection{Wii Remote and Motion Plus Hardware}
The original Wii Remote features motion tracking for vertical movement, left-right horizontal movement, and horizontal rotation through the use of an ADXL330 accelerometer.\cite{wiiAccelerometer}.
In June 2009 Nintendo released the Wii MotionPlus expansion device which contains a dual-axis tuning fork and a single-axis gyroscope\cite{wiiMotionPlus}.
The expansion device improves the motion tracking of the Wii Remote greatly, but makes it larger. Nintendo has now started selling the Wii Remote Plus. It is the same size as the Wii Remote, but has the Wii MotionPlus already built in. Both of the controller types have the ability to provide vibration and basic audio feedback.

%%WRITE SOMETHING ABOUT THE NUNCHUCK?

\subsection{Wii Remote API}
At time of writing no Wii Remote library has been created for the Android OS. Though there are plenty of Wii remote libraries out there none of them are intended to be used on android devices. This section will cover the most developed liberalities that are implemented in Java.

\subsubsection{WiiRemoteJ}
WiiRemoteJ is one of the most complete libraries for the Wii remote. It is a pure Java library with support for a large amount of Wii extensions such as the Wii Guitar, and Wii Balance Board. It does however lack support for the MotionPlus extension. The library has not been update since July 2008. The author has taken down the homepage where the library was originally located, but it can be found on third party websites. \cite{WiiRemoteJ}

\subsubsection{WiiuseJ}
WiiUseJ is a lightweight Java API. It was built on top of the Wiiuse API and only supports the Wii Remote and the Nunchuck. Like the previous library it lacks support for the MotionPlus extension. The project has been discontinued since January 2009. \cite{Wiiusej}

\subsubsection{Motej}
Motej is an open source (licensed under ASL 2.0) library for the Wii remote written in Java. Motej supports only the Wii Remote and IR Camera in it's basic form, but the extras library adds support for the balance board, classic controller, and nunchuk. The project is currently at version 0.9, but was discontinued in 2009. \cite{Motej}

\subsection{Bluetooth}
%General information about bluetooth, and some text about L2CAP and why Android sucks.
Bluetooth is a widely used wireless communication technology for shorter distances. Due to its low power consumption and low cost is has become one of the leading standards in its field and is supported by most modern operating systems, either through integrated hardware or through portable Bluetooth adapter. Most, if not all, modern phones come with a built-in Bluetooth card/radio. %CITATION BITCHES! 
The Wii Remote uses Bluetooth to communicate wirelessly with other devices, using the logical link control and adaptation protocol (L2CAP). 


\chapter{The Case}
%Development
Since no library with Android support currently exist it was decided to use the Motej library as a core for further development.

\section{Introduction}
At the time of writing no research has been done on using motion game controllers as peripherals in a smart-phone based bio-feedback system. In order to answer the research questions posed in this report it will be necessary to create a simple but working prototype of the aforementioned system. The equipment used to realize the system will be a rooted HTC Desire HD \cite{desireHdSpecs} running CyanogenMod 7 \cite{cyanogenMod} and Wii Remotes with the Motion Plus extension. The application will be created using the Android SDK, Java and the Motej library.

%Write about why we chose motej?

\section{Limitations}
%Se på dette avsnittet knut. Tror hele må skriver på nytt hvor avsnittene under er merget og forklart bedre
The majority of Android devices have built-in Bluetooth cards, but the current Android SDK does not offer low level support for the Bluetooth stack, including L2CAP. This constraint can be bypassed on some devices by using reflection to access the socket constructor\cite{l2capHtc}. Access could also be gained through using the Android NDK (Native Develeoper Kit) but this is outside of our scope. Due to L2CAP not being official supported, certain vendors of Android devices have removed the L2CAP protocols completely, meaning that it would be impossible to use their Android OS to connect to the Wii Remote. Therefore the Android OS on the HTC Desire HD was altered to run with CyanogenMod 7 instead of the default HTC Sense.

Motej uses the BlueCove library as a multi-platform interface to the Bluetooth stack. Unfortunately, BlueCove does not support the Android OS. Android comes with its own Bluetooth API, the Android Bluetooth API. This means that the library will have to be re-written to work on the Android OS and support for the MotionPlus will have to be implemented. WiiMoteLib\cite{wiiMoteLib} is a C\# library which has support for the MotionPlus, by looking at the solutions in this library it should reduce the time required to implement MotionPlus support for the Motej library.

\section{Motej on Android}

For Motej to work on Android the BlueCove library has to be replaced with the Android Bluetooth API. This presents a major problem. Wii  remotes uses the low level Bluetooth protocol L2CAP to connect to different platforms. As of Android version 4.1, Jelly Bean, \cite{jellyBean} there is no official support for L2CAP. Android only has full support for the higher level protocol \emph{radio frequency communication} (RFCOMM). RFCOMM is built on top of the L2CAP protocol and provides serial port emulation. 

Though the L2CAP is not directly supported through the Android Bluetooth API it is possible to create an L2CAP socket using a technique called reflection.

\begin{lstlisting}
Class<BluetoothSocket> cls = BluetoothSocket.class;
Constructor<BluetoothSocket> constructor = cls.getDeclaredConstructor(
		int.class, int.class, boolean.class, boolean.class,
		BluetoothDevice.class, int.class, ParcelUuid.class);

int type = 3, fd = -1, port = 0x13;
boolean auth = false, encrypt = false;
// Get some device
BluetoothDevice device = getBluetoothDevice();
ParcelUuid uuid = null;

/* type    - Type of socket (3 for L2CAP)
 * fd      - File descriptor (-1 for new socket)
 * auth    - Require authenticaton
 * encrypt - Require encrypted connection
 * port    - Remote port
 * uuid	   - SDP UUID
 */
// This will crate an L2CAP socket on port 0x13
BluetoothSocket socket = constructor.newInstance(type, fd, auth,
		encrypt, device, port, uuid);
\end{lstlisting}

This method has limitations. Because there is no official support for L2CAP in Android, many vendors roms will throw errors when trying to Bluetooth devices this way. Major vendors such as HTC and Samusng does, as of now, not support L2CAP connections.

\section{Motion Plus Support}
Motej has support for most of the common extensions for the Wii remote. Unfortunately Motion Plus is not one of the supported extensions. Motion Plus was implemented using the already existing extension structure of the Motej library.

Information on how to parse the incoming bytes from the Wii remote was found on the WiiBrew wiki \cite{wiiBrew}. 

%include image of byte meaning here.

\begin{lstlisting}
boolean yawFast = ((extensionData[3] & 0x02) >> 1) == 0;
boolean rollFast = ((extensionData[4] & 0x02) >> 1) == 0;
boolean pitchFast = ((extensionData[3] & 0x01) >> 0) == 0;

float yaw = 
	(extensionData[0] & 0xff | (extensionData[3] & 0xfc) << 6);
float roll = 
	(extensionData[1] & 0xff | (extensionData[4] & 0xfc) << 6);
float pitch = 
	(extensionData[2] & 0xff | (extensionData[5] & 0xfc) << 6);
\end{lstlisting}

An additional class was added for manual calibration of the gyroscope. There is calibration data stored in the Wii remote memory, but WiiBrew states that how to use these data for calibration is still unknown. Some suggestions on how to use the data exist, however the calibrated values are not very accurate compared to the manual calibration method. During manual calibration it is required that the Wii remote is kept still. The calibration takes less than 1 second.

\bibliographystyle{plain}
\bibliography{references}

\end{document}
