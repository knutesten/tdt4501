\documentclass[11pt,twoside,a4paper]{report}
\begin{document}
%Top matter begin
\title{Falldetection}
\author{Knut Esten Melandsø Neks\aa and Dean Lozo}
\date{\today}
\maketitle
% Top matter end

\begin{abstract}

\end{abstract}

\tableofcontents

\chapter{Introduction}
%The Introduction is your thesis in a nutshell. Again, the organization can vary, but a standard introduction includes the following sections:

\section{Motivation}
%Brief description of the research domain and the problem that one wants to address. It should tell the reader why working on this project is worth doing.


\section{Context}
%Very brief description of the context in which the project is done. This is important if you are conducting your work in the context of a wider project.

\section{Research questions}
%What are the questions you are answering with your project? Normally, you specify a main question and related sub-questions. Remember that at the end you have to demonstrate you have answered to the stated questions. It is not uncommon that the questions are changed during the project, but it is important to be as explicit as possible and as early as possible with research questions since they help you to focus.

\section{Research method}
%How the research is conducted. In the previous section you say what you are doing. Here you specify how. The choice of a research method is strictly connected to the type of questions you want to answer.

\section{Results}
%a brief description of the results that you are delivering.

\section{Outline of the report}
%reading guide briefly describing what each chapter of the thesis is about.

\chapter{Problem definition}

\chapter{State of the art}



\end{document}