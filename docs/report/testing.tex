\chapter{Viability Testing}
The following chapter presents factors we believe to be of relevance and important to test so that the viability of the system can be established. The initial section will describe what factors are to be tested, why these are being tested, and what we find acceptable parameters to be. The later section describe the tests performed in order to test each factor and the attained results. All the tests are performed using a Galaxy Nexus \cite{galaxyNexus} and a Wii Remote with a Motion Plus extension unless stated otherwise.

\section{Deciding factors}
The first factor that came to mind was battery life. How was the battery life on the Wii Remote and on the Android smartphone. The majority of Android devices come with specific batteries which are easily rechargeable and have a well documented battery life according to the use. The Wii Remote on the other hand does not. If the battery life is too short it can be cumbersome to change them, come with increase cost and reduce overall satisfaction with the system. We estimated that a Android device battery would at least last 24 hours and be charged while the user is asleep and hence the same rules should apply to the Wii Remote. Therefore a minimum parameter of 24 hours was set in order for the battery life to be satisfactory.

Range is something that should always be considered when dealing with wireless devices. It might be desirable for a user to not carry his phone around if he is at home, or work. Having an awareness of the range would allows the user to comfortably walk around without having to worry about keeping close to the device at all times. Most manufacturers use the Bluetooth Class 2 microchip which allows a minimum range of 10 meters, and thus this is what we consider to be the acceptable value.

The performance of the application on the smartphone needed also to be tested. This was done to ensure that application was smooth and no unexpected behavior would occur on various devices. We found this factor satisfactory if we noticed no hitches or lag from the application in addition that all the functions worked as intended.

We found it necessary to test the initial functionality and tactile feedback that the system provided. The tests were of a qualitative nature and we wanted to see if there were any immediate issues we discovered with the system that could be rectified. We did not perform any major tweaking or spend a large amount of resources finding the optimal position and threshold settings for the system. 

\section{Battery life}
The WiiMote is powered by two AA batteries and battery life is based on several factors. The official response from Nintendo is "up to 30 hours" \cite{wiiBattery}

As the prototype application does not utilize all of the features the Wii Remote offers but only the sensor data and rumble feature we hope to have a battery life of at least 30 hours wit ha fresh set of batteries. The batteries used to perform the test are a fresh set of Duracell Ultra Power batteries. These were chosen because they are available from almost any convenient store. The Galaxy Nexus was connected to power during the entire time, while the WiiMote had a fresh set of batteries put into it before each test.

The application used was a slightly modified version of our prototype application. The changes should have no unintended impact of the battery life of the Wii Remote as most of them were computational and interface data on the Android device, which is not being tested. A timer was added that starts counting when a WiiMote is connected, and stops when the connection is lost. The threshold alarm was disabled, and instead a button allowing manual enabling and disabling of rumbling was added.

Two set of tests were performed in order to measure battery life. The first set involved the WiiMote transmitting sensor data to the Galaxy Nexus. The second test set had the WiiMote rumbling while sensor data was being transmitted. Time and resource constraints limited what type of battery tests could be performed, so two extremes was decided as a good indicator of expected battery life. The results from the tests can be seen in %REFERENCE TO FIGURE

\begin{table}[h]
\centering
\setlength{\extrarowheight}{0,2cm}
\begin{tabular}{p{2cm}|p{4.75cm}|p{4.75cm}|}
\cline{2-3}
&\multicolumn{2}{c|}{\textbf{Time}}\\ \hline
\textbf{Test Nr.} &\textbf{No Rumble} & \textbf{Rumble} \\ \hline
1 & 63:18:04 & 33:12:44 \\ \hline
2 & 61:56:21 & 33:13:45 \\ \hline
3 & 62:23:55 & 33:10:28 \\ \hline
\end{tabular}
\caption{There seemed to be no great variation in the test data, so after 3 runs we concluded that the data was consistent enough.}
\label{}
\end{table}

The table shows a consistent time of around 33 hours with constant rumble and approximately 62 when only sensor data is being transmitted. Our system will have a mix of both depending on the amount of feedback it needs to provide. The expected uptime on the battery type we tested is then somewhere between 33 - 62 hours of use. Considering constant rumbling is highly unlikely, we assume that the battery life is on higher end of the scale.

Our initial expectation was a minimum of 24 hours, the results were far beyond what we expected it to be, which speaks positively for the Wii Remote as possible sensory device. We used brand batteries that boast about longer battery life, but with such a high numbers we believe that even a non-brand and rechargeable battery will be above the 24 hour minimum limit.
\section{Range}
The connection was tested with a 30 meter distance in a open landscape computer lab and no connection issues were present. With a room in between the WiiMote and Android device a range of 15 meters was reached before signs of signal loss started to appear, by this we mean that the 3D object was not updating smoothly or the WiiMote started vibrating franticly.

A distance of 30 meters in an open landscape and 15 meters between walls is well beyond our initial expectation of 10 meters and helps establish the Wii Remote as a viable sensory device for our system.
\section{Performance}
A performance test was performed with the identical software version across three different Android devices. The three device were: A stock HTC One V, stock Google Nexus and a HTC Desire HD using Cynaogenmod 7. The test consisted of connecting the Wii Remote, calibrating, and then changing the orientation of the Wii Remote in several directions at increasing speeds.

\subsection{HTC One S}
As mentioned in the Limitations (ADD REF!?) section our understanding was that the Wii Remote would not be able to establish a connection with an Android device that has HTC Sense on it. The device performed satisfactory within reason, when orientation changed rapidly and franticly it started to shows signs of not being able to keep up with all the rapid alterations.

\subsection{Google Nexus}
The google nexus is considered to be a stock developer phone with no modifications from the re-seller. Surprisingly this one performed the worst out of the test group. At higher speeds stutter became apparent and the sound and vibration started too lag, in addition to the cube not updating. The program would freeze and lag for a second and in the next two seconds it would try to make up for the lost feedback by increasing the frequency of sound and vibration.

\subsection{HTC Desire HD}
This is the oldest model out of the three devices. Like the HTC one S it presented issues with not being able to keep up with the orientation change and reflect it properly in the frequency of the audio and vibration.

\subsection{Performance results}
The fact that the application worked on an Android device with HTC Sense was a major surprise for us, this might mean that devices with Android version 3.0 and 4.0 from HTC do support reflection and allow the application to function properly on the system. The issues that became apparent with the sound and vibration stems from our implementation of the functionality. We focused on getting the intended, and not optimizing performance, we believe that with some time given to optimize this issue can be resolved.

\section{Threshold alarm}
Testing of the threshold alarm were conduced with the writers of this paper as the only test subjects.

A problem we discovered was that the WiiMote vibration was not strong enough for the user to notice that it was vibrating. During the low frequency pulsing vibrations it was nearly impossible to notice the vibrations unless the user was aware it was about to happen. At higher frequency it was slightly better, but still not satisfactory. During the test the WiiMote was facing with the button surface away from the body. The batteries inside the WiiMote are quite large and heavy and we noticed they were soaking most of the vibration and the area around the batteries was barely vibrating. Turning the WiiMote to  so that the surface containing buttons was facing inwards gave slightly better results. We attempted to remove the strap and hold the WiiMote manually against the skin. This did improve the results but did not seem feasible in a real uncontrollable environment.