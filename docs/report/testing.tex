\chapter{Viability Testing}
This section will focus on the qualitative and quantitative tests we perform on the system to help determine if the system is viable as a bio-feedback system, and discover any potential shortcomings with the implementation or hardware. All the tests are performed using a Galaxy Nexus \cite{galaxyNexus} and a Wii Remote with a Motion Plus extension.

\section{Battery life}
Battery life testing was only performed on the WiiMote. Android device batteries are device-specific making it easy to find the estimated battery life, and charging is easier. THe WiiMote is powered by two AA batteries and battery life is based on several factors. The official response from Nintendo is "up to 30 hours" \cite{wiiBattery}

As the prototype application does not utilize all of the features the Wii Remote offers but only the sensor data and rumble feature we hope to have a battery life of at least 30 hours wit ha fresh set of batteries. The batteries used to perform the test are a fresh set of Duracell Ultra Power batteries. These were chosen because they are available from almost any convenient store. The Galaxy Nexus was connected to power during the entire time, while the WiiMote had a fresh set of batteries put into it before each test.

The application used was a slightly modified version of our prototype application. The changes should have no unintended impact of the battery life of the Wii Remote as most of them were computational and interface data on the Android device, which is not being tested. A timer was added that starts counting when a WiiMote is connected, and stops when the connection is lost. The threshold alarm was disabled, and instead a button allowing manual enabling and disabling of rumbling was added.

Two set of tests were performed in order to measure battery life. The first set involved the WiiMote transmitting sensor data to the Galaxy Nexus. The second test set had the WiiMote rumbling while sensor data was being transmitted. Time and resource constraints limited what type of battery tests could be performed, so two extremes was decided as a good indicator of expected battery life. The results from the tests can be seen in %REFERENCE TO FIGURE

\begin{table}[h]
\centering
\setlength{\extrarowheight}{0,2cm}
\begin{tabular}{p{2cm}|p{4.75cm}|p{4.75cm}|}
\cline{2-3}
&\multicolumn{2}{c|}{\textbf{Time}}\\ \hline
\textbf{Test Nr.} &\textbf{No Rumble} & \textbf{Rumble} \\ \hline
1 & 63:18:04 & 33:12:44 \\ \hline
2 & 61:56:21 & 33:13:45 \\ \hline
3 & 62:23:55 & 33:10:28 \\ \hline
\end{tabular}
\caption{There seemed to be no great variation in the test data, so after 3 runs we concluded that the data was consistent enough.}
\label{}
\end{table}

The table shows a consistent time of around 33 hours with constant rumble and approximately 62 when only sensor data is being transmitted. Our system will have a mix of both depending on the amount of feedback it needs to provide. The expected uptime on the battery type we tested is then somewhere between 33 - 62 hours of use. Considering constant rumbling is highly unlikely, we assume that the battery life is on higher end of the scale.

\section{Range}
It should be desirable for users to not have to carry the phone around at home, at work, or if they know they are within a certain vicinity. Hence we decided to an informal in order to test the range of the bluetooth connection once it has been established. The connection was tested with a 30 meter distance in a open landscape computer lab and no connection issues were present. With a room in between the WiiMote and Android device a range of 15 meters was reached before signs of signal loss started to appear, by this we mean that the 3D object was not updating smoothly or the WiiMote started vibrating franticly.

\section{Threshold alarm}
Testing of the threshold alarm was conduced where we were the only test subjects. The tests were of a qualitative nature and we wanted to see if there were any immediate issues we discovered with the system that could be rectified. We did not perform any major tweaking or spend a large amount of resources finding the optimal position and threshold settings for the system. 

A problem we discovered was that the WiiMote vibration was not strong enough for the user to notice that it was vibrating. During the low frequency pulsing vibrations it was nearly impossible to notice the vibrations unless the user was aware it was about to happen. At higher frequency it was slightly better, but still not satisfactory. During the test the WiiMote was facing with the button surface away from the body. The batteries inside the WiiMote are quite large and heavy and we noticed they were soaking most of the vibration and the area around the batteries was barely vibrating. Turning the WiiMote to  so that the surface containing buttons was facing inwards gave slightly better results. We attempted to remove the strap and hold the WiiMote manually against the skin. This did improve the results but did not seem feasible in a real uncontrollable environment.