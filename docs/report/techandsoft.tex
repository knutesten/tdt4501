\chapter{Technology and Software}

\section{Android OS}
Should we write something about the Android OS?

\section{Java}
Is it neccessary to write something aobut Java?

\section{PS3 Move}

\subsection{Motion Controller Hardware}
The PS Move motion controller contains advanced motion sensing, making it an ideal peripheral for a bio-feedback system. It features motion tracking through a three axis accelerometer, a three axis angular rate sensor and provides location tracking through the built in magnetometer. The built in vibration could be used to provide feedback to the user \cite{psMoveTech}.

\subsection{PS Move API}

The PS Move API \cite{PSMoveAPI} is written in C, but contains bindings for various languages, including Java and C\# which are both languages used in smartphone application development. The API explicitly mentions that it runs on Android devices. This is a truth with modifications. It will not run out of the box on an Android device, in addition restructuring and heavy modification of the Android device is required.  The Android OS runs on top of a modified Linux kernel, this kernel does not contain the necessary libraries and drivers in order for the API and Motion controller connectivity to function properly. %Does this need citation?
The next step would be to compile the C API into a shared library using the Android NDK, and use the shared library Java bindings in the Android Java code.

Everything is possible with enough time, but given the time constraint and amount of time required to get this running it was decided that this was not a path worth pursuing for this project. With some much work required it would be simpler to run Ubuntu off an Android device if this becomes available in the future. \cite{ubuntuAndroid}

\section{Wii Remote with Motion Plus}

\subsection{Wii Remote and Motion Plus Hardware}
The original Wii Remote features motion tracking for vertical movement, left-right horizontal movement, and horizontal rotation through the use of an ADXL330 accelerometer.\cite{wiiAccelerometer}.
In June 2009 Nintendo released the Wii MotionPlus expansion device which contains a dual-axis tuning fork and a single-axis gyroscope\cite{wiiMotionPlus}.
The expansion device improves the motion tracking of the Wii Remote greatly, but makes it larger. Nintendo has now started selling the Wii Remote Plus. It is the same size as the Wii Remote, but has the Wii MotionPlus already built in. Both of the controller types have the ability to provide vibration and basic audio feedback.

%%WRITE SOMETHING ABOUT THE NUNCHUCK?

\subsection{Wii Remote API}
At time of writing no Wii Remote library has been created for the Android OS. Though there are plenty of Wii remote libraries out there none of them are intended to be used on android devices. This section will cover the most developed liberalities that are implemented in Java.

\subsubsection{WiiRemoteJ}
WiiRemoteJ is one of the most complete libraries for the Wii remote. It is a pure Java library with support for a large amount of Wii extensions such as the Wii Guitar, and Wii Balance Board. It does however lack support for the MotionPlus extension. The library has not been update since July 2008. The author has taken down the homepage where the library was originally located, but it can be found on third party websites. \cite{WiiRemoteJ}

\subsubsection{WiiuseJ}
WiiUseJ is a lightweight Java API. It was built on top of the Wiiuse API and only supports the Wii Remote and the Nunchuck. Like the previous library it lacks support for the MotionPlus extension. The project has been discontinued since January 2009. \cite{Wiiusej}

\subsubsection{Motej}
Motej is an open source (licensed under ASL 2.0) library for the Wii remote written in Java. Motej supports only the Wii Remote and IR Camera in it's basic form, but the extras library adds support for the balance board, classic controller, and nunchuk. The project is currently at version 0.9, but was discontinued in 2009. \cite{Motej}

\subsection{Bluetooth}
%General information about bluetooth, and some text about L2CAP and why Android sucks.
Bluetooth is a widely used wireless communication technology for shorter distances. Due to its low power consumption and low cost is has become one of the leading standards in its field and is supported by most modern operating systems, either through integrated hardware or through portable Bluetooth adapter. Most, if not all, modern phones come with a built-in Bluetooth card/radio. %CITATION BITCHES! 
The Wii Remote uses Bluetooth to communicate wirelessly with other devices, using the logical link control and adaptation protocol (L2CAP). 
