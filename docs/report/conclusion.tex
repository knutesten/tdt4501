\chapter{Final thoughts}
In the first part of this chapter we discuss our findings and address threats to validity. Secondly, we draw conclusions about our results in respect to the research questions. Finally, we identify further work that needs to be done in order to create a functioning mobile biofeedback system.

\section{Discussion}
In this section we will discuss our findings in relation to the research questions. Threats to validity are also identified and reflected upon.

\subsection{RQ1 - Software Architecture}
Modifiability was one of the main focuses when designing the software architecture. Event-driven architecture was used to to achieve loose coupling between the different components. This gives us the option to replace the Wii Remote with other types of sensors without major changes to the rest of the code. A handler class for the Wii Remote was added to make it easier to implement support for multiple Wii Remotes at a later date. This class also helps separate the Wii Remote components from the rest of the system. 

The Model-View-Controller pattern was utilized in the software architecture to separate the graphical user interface (GUI) from the rest of the system. This assures easy modification of the GUI, which helps ease the creation of a GUI with high usability. Though the current prototype does not contain a complex GUI, usability was seen as an important factor if the prototype application is to be developed further at a later date.

\subsection{RQ2 - Game Controllers}
Initially two different game controllers were considered as possible sensors; these controllers were the Playstation Move and Wii Remote. The Playstation Move Controller is the superior in respect to the sensors, with built in magnetometer and gyroscope. However, it soon became apparent that this controller was not a viable option. This was due to the lack of documentation on the reverse engineering of the controller, as well as the fact that there only exists one API. The reverse engineering of the Wii Remote on the other hand is well documented through multiple communities and a multitude of APIs in different languages. The focus of the project therefore shifted toward supporting the Wii Remote.

There are several Wii Remote libraries written in Java, where Motej was considered the most complete. There was some concern about the fact that all of these Java based libraries were at least three years old, including Motej. Motej was helpful as base for the overall structure as well as a good learning tool for understanding the reverse engineering of the Wii Remote. During the development phase it became apparent how outdated the library actually was. This is why we eventually decided to write our own lightweight library called Motea. One of the reasons for choosing the Wii Remote was the many Java libraries that existed. The amount of work put into modifying Motej and creating Motea was much greater than expected. In retrospect, the Playstation Move should have been inspected further.

\subsection{RQ3 - Limitations}
The prototype developed in this project as well as other applications \cite{iFall, semiSupervisedFallDetection, mobilePhoneBasedFallDetection, detectionOfFalls} have shown that modern smartphones are more than powerful enough to function as a hub for processing sensory data in fall detection and prevention applications. In our project we wanted to explore the possibility of using cheap, off-the-shelf motion based game controllers as the external sensors for the system. 

Madgwick’s AHRS algorithm was used to calculate the orientation of the Wii Remote. The algorithm is run on every sensor event, approximately 100 times per second. Modern smartphones have no problem doing these calculations in real-time, while also rendering the 3D-cube. 

The limitation of the system is the lack of support for reflection used to access the L2CAP Bluetooth protocol which is required by the Wii Remote. Initial research led us to believe that HTC and Samsung devices running stock firmware would not support accessing L2CAP through reflection, and thus leave the application unable to connect to the Wii Remote. To our surprise the application worked fine on the HTC One S during the testing phase. HTC phones running version 4 of the Android OS might therefore be supported, and this needs to be investigated further. Having an application that works out of the box on a major smartphone brand is important for the viability of such an application.

\subsection{RQ4 - Viability}
The viability tests were primarily positive and exceeded the minimum parameters we had set for our system. The Battery life was a positive surprise, the life expectancy was beyond our minimum 24 hours and the ``up to 30 hours’’ stated by Nintendo. Seeing as constant rumbling provided 33 hours and no rumble at all gave approximately 62 hours, it is safe to expect that our application will be above 40 hours as users of the application will ideally make the vibration stop as soon as they can sense it. 

The Wii Remote was connected to the smartphone using the robust and well proven Bluetooth technology for continuous streaming of sensory data. The Class 2 Bluetooth radio used in smartphones and motion controllers gives a theoretical range of 10 meters. Our tests indicates that the actual range is even greater than this. A range of more than 10 meters should be more than satisfactory considering the use of the application.

The feedback test showed that sensing the vibration from the Wii Remote is a problem
The vibration provided by the was barely noticeable when the Wii Remote was strapped to the user. This might be due to the heavy battery absorbing the vibration or the rumble engine simply not being powerful enough. It was necessary to press the Wii Remote firmly against the body before the vibration could clearly be felt. The audio feedback was much easier to interpret even with the minor sound issue on newer Android devices. The implementation of the sound alarm needs further work and testing to assure that it work smoothly on all versions of the Android OS.

\subsection{Threats to validity}
One possible threat to validity is the software architectures focus on modifiability. As a proof-of-concept prototype system, prone to constant changes in hardware and functionality, a highly modifiable architecture is essential. For a system with specified hardware and functionality, modifiability would not be as crucial. In such a system availability and performance would be more important. For a fall detection/prevention application to be used in practice both high availability and performance would have to be guaranteed for obvious reasons. The architectural choices presented in this paper are therefore to be seen from a prototyping point of view, and not a system for production.

Because of the small sample size it is difficult to draw a statistical conclusion with our data in the viability testing. Most of the test were qualitative and based on the intuition and experience of the researchers. Range and performance needs testing under controlled environments before a final conclusion can be drawn. Though the tests are informal of nature, they do give an indicator on how viable the system is as a proof of concept prototype. 

\section{Conclusion}
This project has shown that an Android smartphone and a Wii Remote game controller can be utilized to create a mobile biofeedback system. The initial viability tests show that performance and range is satisfactory, though further work and research is needed to create a system that is effective in practical situations.

\textit{What architectural decision should be considered in a mobile biofeedback system?}

We created a software architecture with focus on modifiability and usability. Modifiability was important to ensure that different components (e.g. sensor choice) could be replaced or modified without affecting the system as a whole. Though little effort was put into the graphical user interface in this project, usability was important for the prototype code to be reused in future work. 

\textit{Which game controllers are viable options in the creation of a mobile biofeedback system?}

Lack of documentation of the reverse engineering of the Playstation Move controller made us conclude that it was not yet a viable option. The Wii Remote has a large community and the reverse engineering is well documented. Wii Remote was therefore chosen as the best option for the system.

\textit{What are the limitations of using the Android OS and game controllers in the creation of a biofeedback system?}

Older smartphones running stock firmware from major Android brands such as HTC and Samsung cannot run the application due to the lack of support for Bluetooth L2CAP protocol. It was discovered that an HTC smartphone running version 4.0 of the stock firmware was able to run the application. Further work should be done in documenting which brands supports the L2CAP protocol.


\textit{What are important factors that will determine if the system has any viability as a mobile biofeedback system, and what are acceptable parameters for these factors?}


We concluded that the most important factors for an initial prototype are battery life, range, performance, and feedback. Battery life, range and performance were all considered satisfactory. The vibration feedback was not satisfactory and was perceived as too weak to be noticeable, but the audio worked well.

\section{Further work}
Still, there is a lot of work to be done to create a functioning fall detection/prevention system using Android smartphones and Wii Remotes. First, the application needs to provide helpful feedback reducing the risk of falls, and alerting about falls. Second, the architecture needs to be tested in respect to availability and performance. Third, a friendly user interface has to be created in order to allow older people, often unexperienced with modern technology, to use the application efficiently and safely. Fourth, an experiment has to be conducted to evaluate the effectiveness of the system and establish the best number of sensors and their optimal position. Finally, the system needs to be tested on different stock firmware to document which brands will run the application out of the box.